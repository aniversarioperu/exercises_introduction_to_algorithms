\begin{description}
 \item[2.2-2] Selection sort.
\end{description}

\begin{lstlisting}
for i in 1:n-1
    key = A[i]
    j = i + 1
    for v in j:n
        if A[v] < key:
            A[j] = key
            key = A[v]
    A[i] = key
    i += 1
\end{lstlisting}

\begin{description}
 \item (i) Loop invariant: elements in the subarray $A[1..i-1]$ will be ordered in ascendent order.
 \item (ii) the last two elements will be ordered anyway so it is only necessary to run to $n-1$.
 \item (iii) best case run scenario $\Theta$$(n^2)$. Worst case scenario $\Theta$$(n^2)$.
\end{description}

\begin{description}
 \item[2.2-3] (i) Linear search of exercise 2.1-3. On average it will run for $\frac{(n+1)}{2}$
 \item (ii) Worst case $n$.
 \item (iii) In the best case scenario the first item will be the one that is needed $\Theta$$(1)$. In then
       worst case scenario the item will be the last of the array $\Theta$$(n)$.
\end{description}

\begin{description}
 \item[2.2-4] Modification of algorithms for best case scenarios. For sorting algorithms, check if the array
 is already sorted. For search algorithm, check if the first item returns the value that we are looking for.
\end{description}